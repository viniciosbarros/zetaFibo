%
%
%	Vinicios Barros - 2013
%	Template simples pra escrita rápida em portugues
%
%
%

\documentclass[10pt,a4paper]{article}
	\usepackage[brazil]{babel} % em portugues brasileiro
	\usepackage[utf8]{inputenc}
	\usepackage[T1]{fontenc}
	\usepackage{tgschola}
	\usepackage{fancyvrb}
	\usepackage{graphicx}
	\usepackage{tabularx}
	\usepackage[usenames,dvipsnames]{color}

\begin{document}

{\centering{}
\section*{\huge{Treinamento Modulo de Software}}
Vinicios Barros \\
}

\section*{1}

{\raggedright{}	 %  Right justified

	\paragraph{}
	O desafio proposto pode ser resolvido em linguagem C
	cumprindo o requisito de ser uma solução para 
	Zeta-Fibonacci (ZF), ou seja, encontrar	a série cujo 
	resultado da soma seja o valor de entrada.
	A função ZF deve receber como entrada um valor inteiro
	de 0 a até pelo menos 1000000000 (1 Bilhão), e encontrar
	uma série que seja um subdominio da série de Fibonacci,
	a soma dessa série deve ser o valor solicitado na 
	entrada do programa.\\
	Por exemplo:\\
\begin{Verbatim}[frame=single,
			framesep=3mm,
			xrightmargin=2cm]
	ZF(5) = Fb(4) + Fb(3)
	Fb(4) = 3
	Fb(3) = 2
\end{Verbatim}
	\paragraph{} 
	Tendo esse desafio em vista já sabemos que iremos precisar
	de um algoritmo para a série de Fibonacci que seja
	implementado em C. O Algoritmo mais largamente utilizada 
	para implementar a função de Fibonacci é dado por uma 
	repetição recursiva para ``n'' elementos da série. 
	De forma geral a solução seria: \\


\begin{Verbatim}[frame=single,
			framesep=3mm,
			xrightmargin=2cm]

	fibonacci(n)
	retorna fibonacci(n-2) + fibonacci(n-1)

\end{Verbatim}
}

{\raggedleft{}
{
	\it{*desconsidera os dois primeiros termos}
}

{\raggedright{}

	\paragraph{}
	Utilizando esse algoritmo em algumas execuções podemos notar que 
	gastasse muito tempo para executar a função de Fibonacci (Fb) para
	valores mais altos. Pois a cada execução de Fb ele precisa
	recalcular os mesmos valores da série. Quanto mais alto o 
	valor, mais vezes essa função sera executada. 
	Ou seja, essa solução é muito lenta para nossa necessidade.	
	A idéia para implementar uma solução otimizada para o problema 
	é aplicar programação dinâmica. Através do uso de
	programação dinâmica é possivel evitar o recálculo da série
	de Fibonacci, certo de que a solução final para nosso problema é uma
	sobreposição de subproblemas resolvidos por recurções da função de
	Fibonacci. A solução prévia garante a resolução	de parte do problema.
	Os subproblemas então se sobrepoem e podem compor a solução final. 
	Logo, tendo em mente que a solução é composta por várias recurções
	da função que encontra a série, e que essa função pode ser 
	recursivamente implementada, a sequencia não precisa ser calculada 
	todas as vezes. Para uma certa entrada o valor
	da série será encontrado, e esse será armazenado em uma estrutura
	de dados para posterior acesso.

% Note the 'Verbatin' case sensitiveness and not 'verbatin'
% this uses the fancyvrb package to draw a box around text
\begin{Verbatim}[frame=single,
				framesep=3mm,
				numbers=left,
				xrightmargin=2cm]
fibonacci(n)
{
	int fib;
	if (n == 0)
		return vector[0];
	else {
		fib = vector[n];
		if (fib == 0)
			fib = fibonacci(n - 2) 
			    + fibonacci(n - 1);
		vector[n] = fib;
	}
	return fib;
}
\end{Verbatim}

	\paragraph{}Podemos notar que essa mudança já desempenha um papel 
	significativo no tempo de execução para encontrar os 
	valores, pois essa função sera executada ``n'' vezes também, 
	o objetivo é encontrar uma soma, e não somente o 
	resultado da série.

	\paragraph{}Uma maneira como podemos aumentar o desempenho na busca
	pela soma, é no uso de heurísca. Podemos avaliar olhando 
	para a solução que a soma para o valor de entrada, pode ser quantificado
	de forma a manter uma proximidade maior do máximo valor da série que seja menor
	que a entrada, e a partir desse valor, seja feito a busca pelos valores
	menores que podem compor o resultado. A busca iniciaria nos maiores valores
	da série e percorreria os valores de forma decrescente.
	Isso limita a discrepância de uma busca	em profundidade que levaria 
	a uma quantidade maior de combinações mas aumentaria significativamente
	o tempo de execução da busca. O algoritmo tenta trabalhar 
	com a idéia de heuristica aproximativa, pois fornece a solução
	dentro de um limite de qualidade absoluto. 
	Ainda para enquadrar melhor nosso método, podemos
	dizer que se trata de uma metaheurística, pois o procedimento aplicado
	visa explorar a instância do problema e seu espaço de solução.

\newpage

	\paragraph{} A explicação fica melhor com um exemplo. Caso a entrada
	seja o valor 15, poderiamos calcular a série somente até 15, ou caso
	ele não pertença a série, somente até o próximo valor maior que 15.
	A condição de parada tornasse o valor de entrada ou o próximo valor
	maior que ele.
	
	\begin{Verbatim}[frame=single,
				framesep=3mm,
				xrightmargin=7cm]
  0 1 1 2 3 5 8 13 21
\end{Verbatim}

	\paragraph{} 15 não pertence a série, pois chegamos ao valor 21
	logo após o 13, sendo esse último, o número a partir do qual iniciamos
	as buscas pela soma. E de forma recursiva tentamos encontrar a soma ideal
	ou composições da mesma através da verificação se ainda é possivel adicionarmos
	um novo valor da série a soma. Se essa soma ultrapasa 15 esse valor da série
	é descartado e a busca continua.


{\raggedright{}
	\paragraph{}A saída do programa deve ser:

		\begin{Verbatim}[frame=single,
				framesep=3mm,
				xrightmargin=8cm]
Fb(7)= 13 
Fb(3)= 2 
\end{Verbatim}
	
	\paragraph{}Onde qualquer valor entre 13 e 2, não satisfaz a soma por fornecer
	um resultado superior a 15.

	\paragraph{}Para análisarmos o tempo de execução do programa vamos fazer um teste com 
	o valor ``100000000'' como entrada, o programa efetuará a busca pelos valores da série
	que somados chegem a esse valor, e imprimira a Fb correspondente de cada elemento 
	que compoem a soma. O programa permite a execução de duas formas diferentes de efetuar
	essa busca. Uma utilizando programação dinâmica e a outra não.

\newpage
	\paragraph{}saida sem programação dinâmica:
	\begin{Verbatim}[frame=single,
				framesep=3mm,
				xrightmargin=3cm]
Total Fb time = 5.436326 seconds
Total time = 5.436327 seconds

Fb(39)= 63245986 
Fb(37)= 24157817 
Fb(35)= 9227465 
Fb(32)= 2178309 
Fb(30)= 832040 
Fb(28)= 317811 
Fb(23)= 28657 
Fb(21)= 10946 
Fb(15)= 610 
Fb(13)= 233 
Fb(11)= 89 
Fb(9)= 34 
Fb(4)= 3 
Fb(0)= 0
\end{Verbatim}

}

	\paragraph{}saida com programação dinâmica:
	\begin{Verbatim}[frame=single,
				framesep=3mm,
				xrightmargin=3cm]
./zetaFiboRecursive 100000000
Total Fb time = 0.000002 seconds
Total time = 0.000003 seconds

Fb(39)= 63245986 
Fb(37)= 24157817 
Fb(35)= 9227465 
Fb(32)= 2178309 
Fb(30)= 832040 
Fb(28)= 317811 
Fb(23)= 28657 
Fb(21)= 10946 
Fb(15)= 610 
Fb(13)= 233 
Fb(11)= 89 
Fb(9)= 34 
Fb(4)= 3 
Fb(0)= 0 
\end{Verbatim}

	\paragraph{} Podemos ver que a diferença de tempo de execução do programa
	para esse mesmo valor varia muito, pois sem o uso de programação dinâmica
	a execução pode levar quase 6 segundos, enquanto utilizando mais estrategias,
	melhor heuristica e com o foco no desempenho, nosso programa executa em
	milésimos de segundo. 

	\paragraph{}A próxima saída é só mais uma curiosidade sobre a solução, onde
	além da programação dinâmica, o programa deve apresentar um ganho 
	de desempenho por não utilizar chamadas recursivas da função Fb, a solução aqui
	é o uso somente de iteração ``for''.


\paragraph{}saida com programação dinâmica e busca de Fibonacci baseada em ``for loop'':
	\begin{Verbatim}[frame=single,
				framesep=3mm,
				xrightmargin=3cm]
./zetaFiboNonRecursive 100000000
Fb time = 0.000001 seconds
Total time = 0.000002 seconds

Fb(45)= 1134903170 
Fb(39)= 63245986 
Fb(36)= 14930352 
Fb(32)= 2178309 
Fb(28)= 317811 
Fb(26)= 121393 
Fb(24)= 46368 
Fb(20)= 6765 
Fb(17)= 1597 
Fb(14)= 377 
Fb(10)= 55 
Fb(6)= 8 
Fb(2)= 1 
Fb(0)= 0 
\end{Verbatim}
	
	\paragraph{}
	A terceira saida do programa mostra uma versão
	onde o algoritmo de Fibonacci não utiliza chamadas recursivas de si mesmo,
	a solução é feita através de uma recursão for loop e isso torna o desempenho
	ainda melhor, apesar de ser praticamente imperceptivel, pois o maior
	mérito esta no uso de programação dinâmica e heurística.

	\paragraph{} Como conclusão podemos dizer que programação dinâmica
	pode ser utilizado em situações onde o resultado prévio não muda, onde
	esse resultado é conhecido e definido. Por isso pode ser encontrado
	e utilizado para cada aplicação que tenhamos necessidade. Heurísticas 
	são essenciais para resolver problemas cujos domínios da função são 
	difíceis ou pouco compreendidos. Em um problema onde não se conhece
	qual o melhor caminho para solução, definesse uma função heurística
	que acreditasse levar a essa solução. As regras são baseadas nas
	experiências anteriores, e substituem as buscas algoritmas que encontram
	a solução correta combinando todas as soluções possíveis.
	Heurísticas exigem menos tempo de execução e se aproximam mais
	da maneira como o ser humano raciocina e chega a resolução do problema.
	Essas técnicas devem ser mais estudadas para serem melhor aplicadas 
	e possamos alcançar o nível de desempenho desejado por nossos 
	programas. Devemos sempre buscar métodos para as mais distintas 
	e diversas aplicações garantindo soluções eficientes e confiáveis.

}
}

{\raggedright{}	
{
\section*{2}
	
	Referências: \\
	\begin{verbatim}
	[1] 
	\end{verbatim}
}
\end{document}
